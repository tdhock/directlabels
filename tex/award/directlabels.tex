\documentclass[12pt]{article}

\usepackage{fullpage}
\usepackage{hyperref}
\usepackage{graphicx}

\title{The directlabels package:\\ easily add direct
  labels to R plots}

\author{Toby Dylan Hocking\\
  toby.hocking@inria.fr\\
  \url{http://cbio.ensmp.fr/~thocking}
}

\begin{document}

\maketitle

Recent years have seen the development of sophisticated plotting
systems for multivariate data in R. In particular, the power of the
lattice and ggplot2 systems allows for automatic legend generation for
plots labeled using different colors according to a categorical
variable. This approach of using legends to decode color schemes is
generally easy to implement but often makes for statistical graphics
that are difficult or impossible to decode if there are too many
colors. An easier method of decoding color schemes is by use of direct
labels:

\begin{center}
\includegraphics[width=\textwidth]{figures}
\end{center}

Direct labels are inherently more intuitive to decode than legends,
since they are placed near the related data. However, direct labels
are not widely used because they are often much more difficult to
implement than legends, and their implementation varies between
plotting systems.

The directlabels package solves these problems by providing a simple,
unified interface for direct labeling in R. Given a lattice or ggplot2
plot saved in the variable \texttt{p}, direct labels can be added by
calling \texttt{direct.label(p,f)} where \texttt{f} is a Positioning
Function that describes where labels should be placed as a function of
the data. The power of this system lies in the fact that you can write
your own Positioning Functions, and that any Positioning Function can
be used with any plot. So once you have a library of Positioning
Functions, direct labeling becomes trivial and so can more easily used
as a visualization technique in everyday statistical practice. The
directlabels package comes with several Positioning Functions for
different plot types, and a system of intelligent defaults so that
often you don't even need to specify the Positioning Function.

The package was implemented using S3 methods to provide a unified
interface for labeling lattice and ggplot2 plots, and could be
extended to work with other plotting systems in R. This elegantly
separates of the tasks of label position calculation and label
drawing, thus resulting in a system that allows maximum
customizability, code re-use, and user-friendlyness.

Development was started by Toby Dylan Hocking on 17 July 2009 while he
was working toward his master's degree by doing a research internship
at the French National Agriculture Institute (INRA) at
Jouy-en-Josas. The research project he was working on called for the
use of direct labels in several different contexts, but no general
framework for doing this in R existed at the time. Since then, he has
started doctoral studies in statistical learning with Jean-Philippe
Vert and Francis Bach, which will be completed in 2012.

The package is currently hosted on R-Forge and can be installed
using the following R code:

\begin{verbatim}
install.packages(c("ggplot2","ElemStatLearn","mlmRev"))#dependencies+examples
install.packages("directlabels",repos="http://r-forge.r-project.org")
\end{verbatim}

There are 2 principal sources of documentation for the directlabels
package. First, extensive documentation for all functions is included
using the standard Rd mechanism for R packages. In particular, the
plots shown above can be recreated by executing the R command
\texttt{example(direct.label)} after loading the package with
\texttt{library(directlabels)}. Second, a website hosted by R-Forge
shows several examples that motivate the use of the package:

\url{http://directlabels.r-forge.r-project.org/}

Thank you for your consideration of the directlabels package for the
John M. Chambers Statistical Software Award. The package's ability to
make direct labeling easier will hopefully make clearer, direct
labeled graphics more common in everyday statistical practice.

\end{document}
